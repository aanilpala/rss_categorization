\begin{center} \textbf{\huge Methodology} \end{center}
\textbf{\large Batch and On-line}\\
The batch model implements a classic Traing-Testing-Phase setup. A subset training points are pre-collected from a stream and used to build a final model on  which three testsets are applied. The batch model functions as a reference model to observe the performance of the initial training over time. \\
\textbf{\large Bruteforce}\\
On the other hand On-line Learning will change the model with the arriving of new data points.
The bruteforce approach updates the model after a period of time through retraining. Based on a sliding window over the stream with a constant number of data points 
the model is rebuilt.\\
\textbf{\large Threshold-triggered}\\
As a variation of the bruteforce model-update the threshold triggered one will rebuild the model on a sliding window as soon as the performance of our model is beneth a certain threshold. \\
\textbf{\large Incremental}\\
In the incremental model, unlike the other update models, following an initial training-only phase, all the stream items are used for training right after the prediction. This way the we aimed to achieve a real-time response to the changes in the data stream. Model is updated by 'dampening' the feature vectors by a function of the learning rate used. Initial traing size of 600 and abs(arctan(x)/(pi/2)) as the learning function ,guranteeing learning rate converges to 0 or 1 as the error gets smaller or infinitely high respectively, are used in our implementation. 
